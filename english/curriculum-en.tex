%%%%%%%%%%%%%%%%%
% This is an example CV created using altacv.cls (v1.1.5, 1 December 2018) written by
% Neeraj Giri (giri492neeraj@gmail.com), based on the
% Cv created by BusinessInsider at http://www.businessinsider.my/a-sample-resume-for-marissa-mayer-2016-7/?r=US&IR=T
%
%% It may be distributed and/or modified under the
%% conditions of the LaTeX Project Public License, either version 1.3
%% of this license or (at your option) any later version.
%% The latest version of this license is in
%%    http://www.latex-project.org/lppl.txt
%% and version 1.3 or later is part of all distributions of LaTeX
%% version 2003/12/01 or later.
%%%%%%%%%%%%%%%%

%% If you are using \orcid or academicons
%% icons, make sure you have the academicons
%% option here, and compile with XeLaTeX
%% or LuaLaTeX.
% \documentclass[10pt,a4paper,academicons]{altacv}

%% Use the "normalphoto" option if you want a normal photo instead of cropped to a circle
% \documentclass[10pt,a4paper,normalphoto]{altacv}

\documentclass[10pt,a4paper,ragged2e]{altacv}

%% AltaCV uses the fontawesome and academicon fonts
%% and packages.
%% See texdoc.net/pkg/fontawecome and http://texdoc.net/pkg/academicons for full list of symbols. You MUST compile with XeLaTeX or LuaLaTeX if you want to use academicons.

% Change the page layout if you need to
\geometry{left=1cm,right=9cm,marginparwidth=6.8cm,marginparsep=1.2cm,top=1.25cm,bottom=1.25cm}

% Change the font if you want to, depending on whether
% you're using pdflatex or xelatex/lualatex
\ifxetexorluatex%
  % If using xelatex or lualatex:
  \setmainfont{Carlito}
\else
  % If using pdflatex:
  \usepackage[utf8]{inputenc}
  \usepackage[T1]{fontenc}
  \usepackage[default]{lato}
\fi

% Change the colours if you want to
\definecolor{StPatricksBlue}{HTML}{1F3D77}
\definecolor{SlateGrey}{HTML}{2E2E2E}
\definecolor{LightGrey}{HTML}{333333}
\colorlet{heading}{StPatricksBlue}
\colorlet{accent}{StPatricksBlue}
	\colorlet{emphasis}{SlateGrey}
\colorlet{body}{LightGrey}

% Change the bullets for itemize and rating marker
% for \cvskill if you want to
\renewcommand{\itemmarker}{{\small\textbullet}}
\renewcommand{\ratingmarker}{\faCircle}

%% sample.bib contains your publications
\addbibresource{publications-en.bib}

\begin{document}
\name{Guilherme I. Gonçalves}
\tagline{Senior Full-Stack Software Engineer}
% Cropped to square from https://en.wikipedia.org/wiki/Marissa_Mayer#/media/File:Marissa_Mayer_May_2014_(cropped).jpg, CC-BY 2.0
\photo{2.5cm}{profile}
\personalinfo{%
  % Not all of these are required!
  % You can add your own with \printinfo{symbol}{detail}
	\email{ginaciog@cern.ch}
	\phone{+55 22 99923 1446}
	\location{Nova Friburgo, Brazil}

	\linkedin{linkedin.com/in/inacioguilherme}
	\github{github.com/ingoncalves} 
	\skype{guilherme.i.g.} 

   %\orcid{orcid.org/0000-0002-6807-3172} % Obviously making this up too. If you want to use this field (and also other academicons symbols), add "academicons" option to \documentclass{altacv}
}

%% Make the header extend all the way to the right, if you want.
\begin{fullwidth}
\makecvheader%
\end{fullwidth}

%% Depending on your tastes, you may want to make fonts of itemize environments slightly smaller
\AtBeginEnvironment{itemize}{\small}

%% Provide the file name containing the sidebar contents as an optional parameter to \cvsection.
%% You can always just use \marginpar{...} if you do
%% not need to align the top of the contents to any
%% \cvsection title in the "main" bar.
\cvsection[page1sidebar-en]{Experience}

\cvevent{Senior Full-Stack Software Engineer}{Prime IT}{Feb 2022 -- Present}{Remote -- Lisbon, Portugal}
\begin{itemize}
	\item Works as a Full-Stack Software Engineer contractor at Siteimprove, a global Software-as-a-Service (SaaS) company, contributing to the design and evolution of large-scale, cloud-based software systems.
	\item Designs, implements, and maintains distributed back-end services in \textbf{Java (Spring Boot)}, as well as in \textbf{C\# (ASP.NET)} and \textbf{Node.js (Express)}, exposing RESTful APIs consumed by multiple client applications.
	\item Develops front-end applications and shared libraries using \textbf{JavaScript/TypeScript} and \textbf{React}, with strong focus on code quality, reusability, and long-term maintainability.
	\item Leads the development and maintenance of the \textbf{Fancy Design System}, a scalable \textbf{React}-based component library, ensuring UI consistency, accessibility compliance (\textbf{WAI-ARIA}), and reliable integration across several products.
	\item Contributes to the architecture and operation of cloud-native systems deployed on \textbf{AWS}, working with services such as \textbf{DynamoDB}, \textbf{CloudFront}, \textbf{ElastiCache}, and \textbf{Redis}, with emphasis on scalability, performance, and reliability.
	\item Works with both \textbf{SQL} and \textbf{NoSQL} data stores, including \textbf{Oracle (SQL)} and \textbf{AWS DynamoDB}, selecting appropriate persistence strategies according to system requirements.
	\item Implements authentication and authorization mechanisms based on \textbf{OAuth 2.0} and \textbf{OpenID Connect} for distributed services and web applications.
	\item Actively participates in agile development within international teams, contributing to technical discussions, code reviews, sprint planning, and continuous improvement following the \textbf{SCRUM} framework.
\end{itemize}

\divider%

%\cvevent{Data Quality Validator}{CERN -- European Organization for Nuclear Research}{Feb 2021 -- Mar 2021}{Remote}
%\begin{itemize}
	%\item Analysis of data quality and integrity of electronic components in the Tile Calorimeter, the hadron calorimeter covering the central region of the ATLAS experiment at the Large Hadron Collider.
%\end{itemize}

%\divider%

\cvevent{User \& Scientific Researcher}{CERN -- European Organization for Nuclear Research}{Mar 2019 -- Present}{Geneva, Switzerland}
\begin{itemize}
    \item Worked on-site from 2019 to 2020 designing and implementing advanced \textbf{energy reconstruction algorithms} for the ATLAS Tile Calorimeter within the \textbf{Athena} software framework, targeting high pile-up conditions at the \textbf{Large Hadron Collider}.
    \item Contributed to the maintenance and upgrade of the Tile Muon Digitizer Board software during the \textbf{Long Shutdown 2} (LS2), ensuring compatibility with upgraded detector electronics.
    \item Returned on-site in 2021 for a short mission to design and implement a \textbf{machine learning}-based energy estimation tool using \textbf{C++} and \textbf{Python}, fully integrated into Athena workflows.
    \item Worked remotely as a \textbf{Data Quality Validator}, analyzing detector data integrity and software reconstruction outputs for the Tile Calorimeter.
    \item Designed and developed a pulse generator and signal simulation library in \textbf{C++} with \textbf{Python} bindings, enabling realistic modeling of calorimeter electronics for physics studies and algorithm development; the tool has been adopted by multiple international research groups.
    \item Contributed from 2023 to 2024 to an authorship qualification task focused on \textbf{linear} and \textbf{machine-learning}-based energy reconstruction algorithms for the upgraded Tile Calorimeter electronics targeting the \textbf{High-Luminosity LHC} (HL-LHC).
	\item Achieved \textbf{full authorship} status within the ATLAS Collaboration in 2024.
\end{itemize}

\clearpage
\cvsection[page2sidebar-en]{Experience}

\cvevent{Technical Lead \& Senior Full-Stack Software Engineer}{O2 Filmes}{Sept 2018 -- Jan 2022}{Remote -- São Paulo, Brazil}
\begin{itemize}
	\item Led the design and development of a collaborative online text editor and project management platform dedicated to screenwriters.
	\item Architected and implemented back-end services using \textbf{Ruby on Rails}, \textbf{MongoDB}, and \textbf{Node.js}, and front-end applications using \textbf{React} and \textbf{Etherpad}, following a \textbf{microservices}-based, cloud-native architecture.
	\item Designed infrastructure as code using \textbf{Terraform}, deploying containerized services with \textbf{Docker} on \textbf{Amazon ECS} clusters.
	\item Took ownership of the full software lifecycle, including system design, implementation, automated testing, deployment pipelines, and operational monitoring.
	\item Acted as technical lead by mentoring developers, defining architectural decisions, prioritizing backlog items, and coordinating development iterations and A/B experiments.
	\item Implemented advanced features such as real-time multi-user collaboration, data visualization with \textbf{D3.js}, notification systems, file import/export pipelines, and complex UI interactions.
\end{itemize}

\divider%

\cvevent{Full-Stack Software Engineer}{DataHex Computer Technology}{May 2016 -- Jun 2018}{Nova Friburgo, Brazil}
\begin{itemize}
	\item Designed and developed an event ticketing application for \textbf{Android} using \textbf{Java}, including data synchronization and Bluetooth-based receipt printing.
	\item Designed and implemented a \textbf{cloud-based} Point of Sale system with business management features, using \textbf{Node.js} for back-end services and \textbf{AngularJS} for front-end applications, following a \textbf{microservices} architecture.
	\item Developed a cross-platform desktop application using \textbf{TypeScript} and \textbf{Electron}, supporting data synchronization over the internet and across multiple local network nodes using \textbf{peer-to-peer} communication.
	\item Implemented a centralized authentication and authorization system based on \textbf{OAuth 2.0} and \textbf{OpenID Connect}.
\end{itemize}

\divider%

\cvevent{Full-Stack Software Engineer \& Mobile Developer}{Vista Group Network}{Jun 2014 -- Apr 2016}{Nova Friburgo, Brazil}
\begin{itemize}
	\item Developed a \textbf{cloud-managed} parking management system using \textbf{Android} devices for ticketing and infringement monitoring, with mobile Android application written in \textbf{Java} and back-end services in \textbf{.NET}.
	\item Developed cross-platform \textbf{iOS} and \textbf{Android} mobile applications for parking ticket purchases using \textbf{Ionic} and \textbf{JavaScript}.
	\item Designed and implemented a computer vision solution to detect free parking spaces using surveillance camera feeds, developed in \textbf{Python} with \textbf{OpenCV}.
\end{itemize}

\cvsection{Open Source Contributions}

\begin{fullwidth}
\cvcontribution{Athena}{The ATLAS Experiment's main offline software}{https://gitlab.cern.ch/atlas/athena/-/merge_requests?state=all&author_username=ginaciog}

\cvcontribution{Etherpad}{A real-time collaborative editor for the web}{https://github.com/ether/etherpad-lite/commits?author=ingoncalves}

\cvcontribution{Scilab}{Free and Open Source software for numerical computation}{https://github.com/scilab/scilab/commits?author=ingoncalves}
\end{fullwidth}

%\cvsection{Publications}

%\nocite{*}

%\printbibliography[heading=pubtype,title={\printinfo{\faBook}{Books}},type=book]

%\divider

%\printbibliography[heading=pubtype,title={\printinfo{\faFileTextO}{Journal Articles}}, type=article]

%\divider%

%\printbibliography[heading=pubtype,title={\printinfo{\faGroup}{Conference Proceedings}},type=inproceedings]
% 
%% If the NEXT page doesn't start with a \cvsection but you'd
%% still like to add a sidebar, then use this command on THIS
%% page to add it. The optional argument lets you pull up the
%% sidebar a bit so that it looks aligned with the top of the
%% main column.
%% \addnextpagesidebar[-1ex]{page3sidebar}


\end{document}
