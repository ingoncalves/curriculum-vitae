%%%%%%%%%%%%%%%%%
% This is an example CV created using altacv.cls (v1.1.5, 1 December 2018) written by
% Neeraj Giri (giri492neeraj@gmail.com), based on the
% Cv created by BusinessInsider at http://www.businessinsider.my/a-sample-resume-for-marissa-mayer-2016-7/?r=US&IR=T
%
%% It may be distributed and/or modified under the
%% conditions of the LaTeX Project Public License, either version 1.3
%% of this license or (at your option) any later version.
%% The latest version of this license is in
%%    http://www.latex-project.org/lppl.txt
%% and version 1.3 or later is part of all distributions of LaTeX
%% version 2003/12/01 or later.
%%%%%%%%%%%%%%%%

%% If you are using \orcid or academicons
%% icons, make sure you have the academicons
%% option here, and compile with XeLaTeX
%% or LuaLaTeX.
% \documentclass[10pt,a4paper,academicons]{altacv}

%% Use the "normalphoto" option if you want a normal photo instead of cropped to a circle
% \documentclass[10pt,a4paper,normalphoto]{altacv}

\documentclass[10pt,a4paper,ragged2e]{altacv}

%% AltaCV uses the fontawesome and academicon fonts
%% and packages.
%% See texdoc.net/pkg/fontawecome and http://texdoc.net/pkg/academicons for full list of symbols. You MUST compile with XeLaTeX or LuaLaTeX if you want to use academicons.

% Change the page layout if you need to
\geometry{left=1cm,right=9cm,marginparwidth=6.8cm,marginparsep=1.2cm,top=1.25cm,bottom=1.25cm}

% Change the font if you want to, depending on whether
% you're using pdflatex or xelatex/lualatex
\ifxetexorluatex
  % If using xelatex or lualatex:
  \setmainfont{Carlito}
\else
  % If using pdflatex:
  \usepackage[utf8]{inputenc}
  \usepackage[T1]{fontenc}
  \usepackage[default]{lato}
\fi

% Change the colours if you want to
\definecolor{StPatricksBlue}{HTML}{1F3D77}
\definecolor{SlateGrey}{HTML}{2E2E2E}
\definecolor{LightGrey}{HTML}{666666}
\colorlet{heading}{StPatricksBlue}
\colorlet{accent}{StPatricksBlue}
\colorlet{emphasis}{SlateGrey}
\colorlet{body}{LightGrey}

% Change the bullets for itemize and rating marker
% for \cvskill if you want to
\renewcommand{\itemmarker}{{\small\textbullet}}
\renewcommand{\ratingmarker}{\faCircle}

%% sample.bib contains your publications
\addbibresource{publications-en.bib}

\begin{document}
\name{Guilherme Gonçalves}
\tagline{Computer Engineer}
% Cropped to square from https://en.wikipedia.org/wiki/Marissa_Mayer#/media/File:Marissa_Mayer_May_2014_(cropped).jpg, CC-BY 2.0
\photo{2.5cm}{profile}
\personalinfo{%
  % Not all of these are required!
  % You can add your own with \printinfo{symbol}{detail}
	\printinfo{\faInfo}{07 June 1994, Brazilian}
	\email{inacio.guilherme@gmail.com}
	\phone{+55 22 99923-1446}
	\location{Nova Friburgo, Brazil}

	\linkedin{linkedin.com/in/inacioguilherme}
	\github{github.com/ingoncalves} 

   %\orcid{orcid.org/0000-0002-6807-3172} % Obviously making this up too. If you want to use this field (and also other academicons symbols), add "academicons" option to \documentclass{altacv}
}

%% Make the header extend all the way to the right, if you want.
\begin{fullwidth}
\makecvheader
\end{fullwidth}

%% Depending on your tastes, you may want to make fonts of itemize environments slightly smaller
\AtBeginEnvironment{itemize}{\small}

%% Provide the file name containing the sidebar contents as an optional parameter to \cvsection.
%% You can always just use \marginpar{...} if you do
%% not need to align the top of the contents to any
%% \cvsection title in the "main" bar.
\cvsection[page1sidebar-en]{Experience}

\cvevent{Full-stack Developer \& Software Engineer}{O2 Filmes}{Sept 2018 -- Present}{Remote}
\begin{itemize}
\item Developed a robust and online text editor and project manager dedicated to screenwriters.
\item Developed its Back-end using Ruby on Rails and its Front-end using React.
\item Implemented features such as multi-user tracking, data visualization, notifications, file upload, advanced graphical interface features, etc.
\end{itemize}

\divider

\cvevent{Scientific Researcher \& Software Engineer}{CERN -- European Organization for Nuclear Research}{Mar 2019 -- Mar 2020}{Geneva, Switzerland}
\begin{itemize}
	\item Designed and developed an energy estimation algorithm for the ATLAS Tile Calorimeter.
	\item The tool is based on machine learning techniques and was developed using C++ in a world-wide distributed system.
	\item Created a pulse generator used to simulate electronic readouts for data processing and physics analysis.
\end{itemize}

\divider

\cvevent{Full-stack Developer \& Software Engineer}{DataHex Computer Technology}{May 2016 -- Jun 2018}{Nova Friburgo, Brazil}
\begin{itemize}
\item Developed and Designed an Event Ticket application for Android with data synchronization and Bluetooth printer.
\item Developed and Designed a cloud-based Point of Sale system with business management features.
\item Developed its Back-end using NodeJS and its Front-end using AngularJS following the micro-services approach.
\item Developed a desktop application using Electron with data synchronization across the internet and among multiple local network nodes (using P2P).
\item Developed an industry-standard authentication system implementing the OAuth 2.0 and OpenID Connect protocols.
\end{itemize}

\divider

\cvevent{Full-stack Developer \& Mobile Developer}{Vista Group Network}{Jun 2014 -- Apr 2016}{Nova Friburgo, Brazil}
\begin{itemize}
\item Developed a cloud-managed parking software using Android devices for ticketing and infringement monitoring.  
\item Developed an iOS and Android application for parking tickets purchase using Ionic and VB-Net.
\item Created software to detect free parking spaces using image processing from surveillance cameras. The tool was developed using Python and Open-CV.
\end{itemize}

\clearpage

\cvsection[page2sidebar-en]{Publications}

\nocite{*}

%\printbibliography[heading=pubtype,title={\printinfo{\faBook}{Books}},type=book]

%\divider

\printbibliography[heading=pubtype,title={\printinfo{\faFileTextO}{Journal Articles}}, type=article]

\divider

\printbibliography[heading=pubtype,title={\printinfo{\faGroup}{Conference Proceedings}},type=inproceedings]
% 
%% If the NEXT page doesn't start with a \cvsection but you'd
%% still like to add a sidebar, then use this command on THIS
%% page to add it. The optional argument lets you pull up the
%% sidebar a bit so that it looks aligned with the top of the
%% main column.
%% \addnextpagesidebar[-1ex]{page3sidebar}


\end{document}
